\documentclass[12pt, preprint]{aastex}

\usepackage{subfigure}
\usepackage{color}
\usepackage{hyperref}
\usepackage{url}
\usepackage{natbib}

\newcommand{\project}[1]{\textsl{#1}} 
\newcommand{\cpm}{\project{CPM}}
\newcommand{\todo}[1]{\textbf{#1}}

\bibliographystyle{apj}
\definecolor{linkcolor}{rgb}{0,0,0.5}
\hypersetup{colorlinks=true,linkcolor=linkcolor,citecolor=linkcolor,
            filecolor=linkcolor,urlcolor=linkcolor}

\begin{document}

\title{\cpm\ Difference imaging}
\author{}

\section{Difference imaging}
Difference imaging or image subtraction is a method that calculates difference between image frames to detect variable objects. 
The first attempt of difference imaging or image subtraction was made by \cite{imagesub1}, who suggested to calculate a convolution kernel by using a bright star to match different image frames, so that the images can be differenced. 
\cite{alard} improved the method by decomposing the kernel into three gaussians with different of variances and then fitting a constant convolution kernel to match the PSFs of images.
The current preference for difference imaging \citep{varyingkernel} is to divide images into sub-areas and fit a varying kernel to account for the spatial variation of the PSF. This method is implemented and widely used as \project{HOTPANTS}\footnote{\url{http://www.astro.washington.edu/users/becker/v2.0/hotpants.html}}, \project{ISIS}\footnote{\url{http://www2.iap.fr/users/alard/package.html}}.


\clearpage
\bibliography{cdi}
\clearpage

\end{document}